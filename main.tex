\documentclass{beamer}
\usetheme{Madrid}
\usecolortheme{default}
\usepackage[utf8]{inputenc}
\usepackage[turkish]{babel}


\title{IEEE İTÜ RAS Git Eğitimi}
\author{Kadir Yavuz Kurt}
\institute{IEEE İTÜ RAS}
\date{\today}

\begin{document}

% Başlık slaytı
\begin{frame}
    \titlepage
\end{frame}

% İçindekiler slaytı
\begin{frame}{İçindekiler}
    \tableofcontents
\end{frame}

% Bölüm 1: Git Nedir?
\section{Git Nedir?}
\begin{frame}{Git Nedir?}
    \begin{itemize}
        \item Git, bir sürüm kontrol sistemidir.
        \item Kod değişikliklerini kaydetmek, izlemek ve yönetmek için kullanılır.
        \item Dağıtık bir yapıya sahiptir.
        \item Linus Torvalds tarafından 2005 yılında geliştirildi.
    \end{itemize}
\begin{figure}
    \centering
    \includegraphics[width=0.25\linewidth]{img/git_logo.png}
\end{figure}
\end{frame}

% Bölüm 2: Git'i Neden Kullanmalıyız?
\section{Git'i Neden Kullanmalıyız?}
\begin{frame}{Git'i Neden Kullanmalıyız?}
    \begin{itemize}
        \item Ekip projelerinde sürüm kontrolü sağlar.
        \item Değişiklikleri geri almayı kolaylaştırır.
        \item Farklı sürümler üzerinde çalışmayı destekler.
        \item İş birliğini artırır.
    \end{itemize}
\end{frame}

% Bölüm 3: Git'in Temel Komutları
\section{Git'in Temel Komutları}
\begin{frame}{Git'in Temel Komutları}
    \begin{itemize}
        \item `git init`: Yeni bir Git deposu başlatır.
        \item `git add <dosya>`: Dosyada yapılan değişiklikleri hazırlık alanına ekler.
        \item `git commit -m "mesaj"`: Değişiklikleri depoya kaydeder.
        \item `git status`: Depodaki durumu kontrol eder.
        \item `git log`: Commit geçmişini gösterir.
        \item `git remote add`: Ağdaki (internet, ssh, vs.) bir git reposuna yerel repoyu bağlar.
    \end{itemize}
\end{frame}

% Bölüm 4: Pratik: Git Deposu Oluşturma
\section{Pratik: Git Deposu Oluşturma}
\begin{frame}{Pratik: Git Deposu Oluşturma}
    \begin{enumerate}
        \item Yeni bir klasör oluşturun: \texttt{mkdir proje}
        \item Git deposu başlatın: \texttt{git init}
        \item Bir dosya oluşturun: \texttt{touch dosya.txt}
        \item Dosyayı ekleyin: \texttt{git add dosya.txt}
        \item Commit yapın: \texttt{git commit -m "İlk commit"}
    \end{enumerate}
\end{frame}

% Bölüm 5: Git Hosting Hizmetlerine Giriş
\section{Git Hosting Hizmetlerine Giriş}
\begin{frame}{Git Hosting Hizmetlerine Giriş}
    \begin{itemize}
        \item GitHub, GitLab, Bitbucket, Azure DevOps gibi platformlar.
        \item Merkezi depolama ve iş birliği imkanı.
        \item CI/CD (Continuous Integration/Continuous Delivery), proje yönetimi araçları gibi ek özellikler sunar.
    \end{itemize}

\end{frame}

% Bölüm 6: GitHub'a Detaylı Giriş
\section{GitHub'a Detaylı Giriş}
\begin{frame}{GitHub Nedir?}
    \begin{itemize}
        \item GitHub, Git depolarını barındıran bir platformdur.
        \item Ekip projelerinde iş birliğini kolaylaştırır.
        \item Açık kaynak projeleri paylaşmak ve katkıda bulunmak için kullanılır.
    \end{itemize}

\end{frame}

% Bölüm 7: Diğer Git Hizmetleri
\section{Diğer Git Hizmetleri}
\begin{frame}{Diğer Git Hizmetleri}
    \begin{itemize}
        \item \textbf{GitLab}: Git tabanlı bir sürüm kontrolü platformu. CI/CD, issue tracking ve daha fazlasını sunar.
        \item \textbf{Bitbucket}: Atlassian tarafından geliştirilen bir Git ve Mercurial deposu barındırma hizmeti.
        \item \textbf{Azure DevOps}: Microsoft'un sunduğu geliştirme hizmetleri, Git depoları, CI/CD boru hatları ve proje yönetimi araçları içerir.
        \item \textbf{SourceForge}: Açık kaynak projeleri için barındırma ve dağıtım platformu.
    \end{itemize}
    \begin{figure}
        \centering
        \includegraphics[width=0.5\linewidth]{img/git_hizmetleri.png}
    \end{figure}
\end{frame}

% Bölüm 8: GitHub Uygulamaları
\section{GitHub Uygulamaları}
\begin{frame}{GitHub Uygulamaları}
    \begin{itemize}
        \item \textbf{GitHub Desktop}: GUI tabanlı bir Git istemcisidir. Komut satırı kullanmak istemeyenler için idealdir.
        \item \textbf{GitHub Actions}: Sürekli entegrasyon ve dağıtım (CI/CD) süreçlerini otomatikleştirir.
        \item \textbf{GitHub Pages}: Projelerinizi doğrudan GitHub üzerinden barındırıp yayınlamanıza olanak tanır.
        \item \textbf{GitHub Issues}: Proje yönetimi ve hata takibi için kullanılır.
    \end{itemize}
\end{frame}

% Bölüm 9: Git İle Çalışma Araçları
\section{Diğer Git ile Çalışma Araçları}
\begin{frame}{Git ile Çalışma Araçları}
    \begin{itemize}
        \item \textbf{Git GUI}: Git'in resmi GUI istemcisi.
        \item \textbf{SourceTree}: Atlassian tarafından geliştirilen ücretsiz bir Git ve Mercurial istemcisi.
        \item \textbf{Visual Studio Code}: Entegre Git desteği ile popüler bir kod editörü.
        \item \textbf{TortoiseGit}: Windows için bir Git istemcisi, powershell entegrasyonu sağlar.
    \end{itemize}
\end{frame}

% Bölüm 10: GitHub Desktop Kullanımı
\section{GitHub Desktop Kullanımı}
\begin{frame}{GitHub Desktop Kullanımı}
    \begin{enumerate}
        \item Git hizmetlerini terminal üzerinden kullanmak istemiyorsanız masaüstü uygulaması kullanabilirsiniz.
        \item GitHub Desktop'u resmi web sitesinden (https://desktop.github.com/) indirin ve kurun.
        \item GitHub hesabınızla oturum açın.
        \item Yeni bir depo oluşturun veya mevcut bir depoyu klonlayın.
        \item Değişiklikleri yapın ve "Commit to main" butonuna tıklayarak commit yapın.
        \item "Push origin" butonuyla değişiklikleri GitHub'a gönderin.
    \end{enumerate}
\end{frame}

% Bölüm 11: GitHub ile Proje Yönetimi
\section{GitHub ile Proje Yönetimi}
\begin{frame}{GitHub ile Proje Yönetimi}
    \begin{itemize}
        \item \textbf{Branches (Dallar)}: Farklı özellikler veya düzeltmeler üzerinde çalışmak için kullanılır.
        \item \textbf{Pull Requests}: Değişikliklerin gözden geçirilmesi ve ana dal ile birleştirilmesi süreci.
        \item \textbf{Issues}: Hatalar, özellik istekleri ve diğer görevlerin takibi.
        \item \textbf{Projects}: Kanban panoları ile proje yönetimi.
    \end{itemize}
\end{frame}

\section{Uygulama}
\begin{frame}{Uygulama}
    \centering
    \Huge Linux Terminal veya VSCode Terminal Kullanarak Github Reposuna İçerik Yüklemek.

\end{frame}

\section{Son}
\begin{frame}{Sorular?}
    \centering
    \Huge Sorularınız var mı?
\end{frame}

\begin{frame}{Automosphere 24}
\begin{figure}
    \centering
    \includegraphics[width=0.5\linewidth]{img/automosphere24.png}
    \caption{GitHub Automosphere Sponsorluk Linki}
    \label{fig:enter-label}
\end{figure}
\end{frame}






\begin{frame}
    \centering
    \Huge Dinlediğiniz İçin Teşekkürler!
\end{frame}

\end{document}
